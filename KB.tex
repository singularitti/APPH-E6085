% !TEX root = review.tex

\subsubsection{Kleinman--Bylander form of pseudopotential}
As discussed before,
HSC proposed a model pesudopotential, which is the or bare ion pesudopotential, written as
\begin{equation}\label{eq:hsc-pp}
	V_l(\bm{r}) = \mathcal{V} (r) + \sum_{l m} \ket{Y_{lm}} \delta V_l \bra{Y_{lm}},
\end{equation}
where $\mathcal{V} (r)$ is a pure local potential, potential
$V_l^\text{SL} = \ket{Y_{lm}} \delta V_l \bra{Y_{lm}}$ is a semilocal
\footnote{
	semilocal: local in radial coordinate $r$ but nonlocal in angular coordinates $\theta$ and $\varphi$.
}
one but this involves $m n (n+1) / 2$ individual integrals for $m$ points
with $n$ wave vectors in the Brillouin zone.
HSC require that for $r > R_c$, the pesudopotential
$V (r) \rightarrow - \frac{ Z }{ r }$, and thus
$\delta V_l (r) \rightarrow 0 $ for $r > R_c$,
which makes the need to calculate true wave-functions is minimized.
K and B \cite{Kleinman:1982cx} noticed tried to rewrite it into a pure local potential and a pure
nonlocal one. First they considered the relativistic form of $\eqref{eq:hsc-pp}$,
i.e.,
\begin{equation}
	V_j^\text{SL} (\bm{r}) = \mathcal{V} (r) + \sum_{lm} \ket{Y_{lm}} \delta V_j \bra{Y_{lm}},
\end{equation}
where $j = l \pm \frac{ 1 }{ 2 }$
\footnote{
	In non-relativistic approximation, $j = l \pm \frac{ 1 }{ 2 }$ is replaced by $l$.
}
, assume pesudo-wave-function
$\Phi_j (\bm{r}) = \phi(r) Y_{lm} (\theta, \varphi)$,
then replace the semilocal potential $\delta V_j^\text{SL}$ with a pure nonlocal potential
\begin{equation}\label{eq:nl-pp}
	\delta V_j^\text{NL} (\bm{r}, \bm{r}')
	= \sum_{lm} \frac{ \ket{\delta V_j \Phi_{jm}}  \bra{\delta V_j \Phi_{jm}} }{
		\braket{ \Phi_{jm} | \delta V_j | \Phi_{jm} } },
\end{equation}
here $\Phi_{jm}$ is the eigenfunction of $V_j^\text{SL}$.
Then we have
\begin{equation}\label{eq:vnl-phi}
	\begin{split}
		\delta V_j^\text{NL} \ket{\Phi_{j'm'}}
		&=
		\sum_{lm} \frac{
			\ket{\delta V_j \Phi_{jm}}  \braket{\delta V_j \Phi_{jm} | \Phi_{j'm'} }
		}{
			\braket{ \Phi_{jm} | \delta V_j | \Phi_{jm} }
		} \\
		&= \ket{ \delta V_{j'} \Phi_{j'm'} },
	\end{split}
\end{equation}
and we also know
\begin{equation}
	\begin{split}
		\delta V_{j}^\text{SL} \ket{ \Phi_{j'm'} }
		&= \ket{Y_{lm}} \delta V_{j'} \braket{Y_{lm} | \Phi_{j'm'} } \\
		&= \ket{Y_{l'm'}} \delta V_{j'} \ket{\phi_{j'}} \\
		&= \ket{ \delta V_{j'} \Phi_{j'm'} },
	\end{split}
\end{equation}
so we conclude
\begin{equation}\label{eq:kb-result}
	\delta V_j^\text{NL} \ket{\Phi_{j'm'}} = \delta V_{j}^\text{SL} \ket{ \Phi_{j'm'} }.
\end{equation}
From $\eqref{eq:kb-result}$ we know that the nonlocal potential constructed by
$\eqref{eq:nl-pp}$ is equivalent to the semilocal potential $\delta V_{j}^\text{SL}$,
because they result in the same eigenvalue. But the number of
plane-wave integrals is reduced to $mn$.

In $\eqref{eq:vnl-phi}$, $\bra{\delta V_j \Phi_{jm}}$ acts as a projector,\cite{martin2004electronic} makes
\begin{equation}
	\braket{\delta V_j \Phi_{jm} | \Phi_{j'm'} } = \int d\bm{r} \delta V_j (r)
	\Phi_{jm} (\bm{r}) \Phi_{j'm'} (\bm{r}).
\end{equation}
This projector
is \emph{local} because $\delta V_j$ is $0$ out side core radius $R_c$,
irrespective of how $\Phi_{jm}$ is.
So generally,\cite{martin2004electronic}
\begin{equation}
	\braket{ \psi_i | \delta V_j^\text{NL} | \psi_k }
	= \sum_{lm}
	\frac{ \braket{\psi_i | \delta V_j \Phi_{jm}}
		\braket{\delta V_j \Phi_{jm}  | \psi_k } }{
		\braket{ \Phi_{jm} | \delta V_j | \Phi_{jm} } },
\end{equation}
here only projection operations are need to be done,
comparing to
\begin{equation}
	\braket{ \psi_i | \delta V_j^\text{SL} | \psi_k }
	= \int d\bm{r} \, \psi_i \delta V_j^\text{SL} \psi_k,
\end{equation}
where whole integral should be done for each pair of $i$ and $j$.

The nonlocal operator $\delta V_j^\text{NL}$ is improved
by \citeauthor{Vanderbilt:1990is}, who introduced ultrasoft pesudopotential, will be discussed in \ref{sssec:us}.