% !TEX root = review.tex

\subsubsection{PAW method}
\label{sssec:paw}

In general, only ground state properties, such as total state energy or
equilibrium geometries can be investigated by DFT.\cite{Enkovaara:2010jd}
The PAW method is formally an all-electron method which provides an exact linear
transformation
$\hat{\mathcal{T}}$
between the smooth pseudo-wave-functions $\psi_n$ and the
all-electron wave-functions $\tilde{\psi}_n$, where $n$ is the band index.
The PAW approximation contains all the information about the nodal structure of
wave-functions near the nuclei, and it is always possible to reconstruct the
all-electron wave-functions from the pseudo-wave-functions.
In solid state physics, it is popular to use plane-wave basis set, while in
quantum chemistry localized orbitals are more often applied.
The former employs fast Fourier transform, and higher accuracy can be achieved
systematically by increasing the cutoff energy. The localized orbital method does
not have this advantage. But it can provide very compact basis set.

The AE partials wave, $\phi_{i, \bm{R}}$, and PS partial wave $\tilde{\phi}_{i, \bm{R}}$,
are constructed by radial wave-function and spherical harmonics
\begin{align}
	\phi_{i, \bm{R}}         & = \phi_{nl} (\bm{r} - \bm{R}) Y_L(\bm{r}),         \\
	\tilde{\phi}_{i, \bm{R}} & = \tilde{\phi}_{nl} (\bm{r} - \bm{R}) Y_L(\bm{r}),
	\label{eq:tildephi}
\end{align}
where $i = (n, l, m)$ and $L = (l, m)$.\cite{Mortensen:2005ep}
It is required that
\begin{equation}
	\phi_{i, \bm{R}} = \tilde{\phi}_{i, \bm{R}} \text{ outside } R_c.
\end{equation}
The pseudo-density $\tilde{n}$ thus also should be the same as AE density
$n$ outside $R_c$.
The valence band function
\begin{equation}
	\begin{rcases}
		\psi_{n, \bm{R}} = \psi_{n, \bm{R}} \\
		\tilde{\psi}_{n, \bm{R}} = \tilde{\psi}_{n, \bm{R}}
	\end{rcases}
	\text{ for } \abs{\bm{r} - \bm{R}} < R_c,
\end{equation}
and since
\begin{equation}
	\psi_n = \tilde{\psi}_n + \sum_{\bm{R}} (\psi_{\bm{R}} - \tilde{\psi}_{\bm{R}}),
\end{equation}
latter term, i.e.,
the correction to pesudo-wave-function,
is nonzero only in $\abs{\bm{r} - \bm{R}} < R_c$
for a certain $\bm{R}$.

Like usual pesudopotential method, there is a cutoff radius for each atom at $R_c$.
In practice, we often choose the smooth projector functions to be
\cite{Mortensen:2005ep,rostgaard2009projector}
\begin{equation}
	\tilde{p}_{i, \bm{R}} = \tilde{p}_{nl} (\bm{r} - \bm{R}) Y_L (\bm{R}),
	\label{eq:tildep}
\end{equation}
and it must satisfy
\begin{equation}
	\braket{ \tilde{p}_{i, \bm{R}} | \tilde{\phi}_{j, \bm{R}} } = \delta_{ij}
	\text{ for } \abs{\bm{r} - \bm{R}} < R_c.
\end{equation}

The core states of the atoms, $\phi_{i}^\text{core}(\bm{r} - \bm{R})$,
are fixed to the reference shape for the isolated atom. This is called the
frozen core approximation.\cite{rostgaard2009projector}


Then we introduce one-center expansion,
\begin{align}
	\psi_{n, \bm{R}}         & = \sum_{\bm{R}} \phi_{i, \bm{R}} P_{i, n} (\bm{r} - \bm{R}), \\
	\tilde{\psi}_{n, \bm{R}} & =
	\sum_{\bm{R}} \tilde{\phi} _{i, \bm{R}} P_{i, n} (\bm{r} - \bm{R}),
\end{align}
that is, they share same set of $\{ P_{i, n} \}$, where
\begin{equation}
	P_{i, n} (\bm{r} - \bm{R}) = \braket{ \tilde{p}_{i, \bm{R}} | \tilde{\psi}_n (\bm{r}) }.
\end{equation}


\begin{equation}\label{eq:T}
	\hat{\mathcal{T}} = 1 + \sum_{\bm{R}} \sum_{i}
	(\ket{\phi_{i, \bm{R}}} - \ket{\tilde{\phi}_{i, \bm{R}}}) \bra{\tilde{p}_i},
\end{equation}


For $r > R_c$, $\phi_{i, \bm{R}} = \tilde{\phi}_{i, \bm{R}}$. Thus $\eqref{eq:T}$ is
$1$, i.e., the rest part of $\hat{\mathcal{T}}$ is $0$ outside augmentation sphere.


In $\eqref{eq:tildephi}$ and $\eqref{eq:tildep}$, how do we choose the radial
wave-functions? Usually,
\begin{align}
	\tilde{\phi}_{nl} & = \sum_{i=0}^{3} c_i r^{2i},                    \\
	\tilde{p}_{nl}    & = \bigg( -\frac{ 1 }{ 2 }\nabla^2 + \tilde{v} -
	\varepsilon_{nl} \bigg) \tilde{\phi}_{nl}.
\end{align}
The coefficients $c_i$ are chosen so that $\tilde{\phi}_{nl}(\bm{r} - \bm{R})$ joins
$\phi_{nl}(\bm{r} - \bm{R})$ smoothly at $r = R_c$.


In practical calculations, it is usually enough to include one or two
projector functions per angular momentum channel.\cite{Enkovaara:2010jd}

