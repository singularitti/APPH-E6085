% !TEX root = review.tex

\subsubsection{Ultrasoft pesudopotential construction}
\label{sssec:us}

\citeauthor{Vanderbilt:1990is} construct a pesudo-wave-function $\ket{ \phi_i }$, and a
pesudopotential $\delta V^\text{NL}$ associated with it,
satisfying for $r \ge R_c$,
\begin{align}
	\ket{ \phi_i }                        & = \ket{ \psi_i }, \\
	\delta V^\text{NL}                    & = 0,              \\
	V &= V^\text{LOC} + \delta V^\text{NL} = \mathcal{V},
\end{align}
where $\mathcal{V}$ is the all-electron potential.
To introduce $\delta V^\text{NL}$, first construct wave-function
\begin{equation}
	\ket{ \chi_i } = \bigg( \varepsilon_i -
	\Big( - \frac{ 1 }{ 2 } \nabla^2 + V^\text{LOC} \Big)
	\bigg) \ket{ \phi_i },
\end{equation}
this wave-function vanishes outside $R_c$ since
\begin{equation}\label{eq:us-ae}
	\bigg( \varepsilon_i -
	\Big( - \frac{ 1 }{ 2 } \nabla^2 + \mathcal{V} \Big)
	\bigg) \ket{ \psi_i } = 0.
\end{equation}
Then the nonlocal operator is
\begin{equation}\label{eq:us-pp}
	\delta V^\text{NL} = \frac{ \ket{\hat{P} \phi_i } \bra{ \hat{P} \phi_i }}{
		\braket{ \phi_i | \hat{P} | \phi_i } }
	=
	\frac{ \ket{ \chi_i } \bra{ \chi_i } }{
		\braket{ \chi_i | \phi_i }
	}
\end{equation}
which has the form of KB form $\eqref{eq:nl-pp}$.
Apply
\begin{equation}
	H =  - \frac{ 1 }{ 2 } \nabla^2 + V^\text{LOC} + \delta V^\text{NL}
\end{equation}
to $\ket{ \phi_i }$ it will yield $\varepsilon_i \ket{ \phi_i }$ directly,
where $\varepsilon_i$ is the all-electron eigenvalue, see $\eqref{eq:us-ae}$.
Thus we know $V^\text{LOC} + \delta V^\text{NL}$ is a validated
pesudopotential.